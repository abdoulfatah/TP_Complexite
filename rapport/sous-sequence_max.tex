% !TEX TS-program = pdflatex
% !TEX encoding = UTF-8 Unicode

% This is a simple template for a LaTeX document using the "article" class.
% See "book", "report", "letter" for other types of document.

\documentclass[11pt]{article} % use larger type; default would be 10pt

\usepackage[utf8]{inputenc} % set input encoding (not needed with XeLaTeX)
%\usepackage[T1]{fontenc}
\usepackage[francais]{babel}

%%% PAGE DIMENSIONS
\usepackage{geometry}
\geometry{a4paper}

\usepackage{graphicx} % support the \includegraphics command and options

% \usepackage[parfill]{parskip} % Activate to begin paragraphs with an empty line rather than an indent

%%% PACKAGES
\usepackage{booktabs} % for much better looking tables
\usepackage{array} % for better arrays (eg matrices) in maths%
\usepackage{verbatim} % adds environment for commenting out blocks of text & for better verbatim
\usepackage{subfig} % make it possible to include more than one captioned figure/table in a single float
% These packages are all incorporated in the memoir class to one degree or another...

%%% HEADERS & FOOTERS
\usepackage{fancyhdr} % This should be set AFTER setting up the page geometry
\pagestyle{fancy} % options: empty , plain , fancy
\renewcommand{\headrulewidth}{0pt} % customise the layout...
\lhead{}\chead{}\rhead{}
\lfoot{}\cfoot{\thepage}\rfoot{}

%%% SECTION TITLE APPEARANCE
\usepackage{sectsty}
\allsectionsfont{\sffamily\mdseries\upshape} % (See the fntguide.pdf for font help)
% (This matches ConTeXt defaults)

%%% ToC (table of contents) APPEARANCE
\usepackage[nottoc,notlof,notlot]{tocbibind} % Put the bibliography in the ToC
\usepackage[titles,subfigure]{tocloft} % Alter the style of the Table of Contents
\renewcommand{\cftsecfont}{\rmfamily\mdseries\upshape}
\renewcommand{\cftsecpagefont}{\rmfamily\mdseries\upshape} % No bold!

\usepackage{algorithm2e}

%%% END Article customizations

%%% The "real" document content comes below...

\title{TP1 - Sous-séquences maximales}
\author{Sébastien Delecraz \and \'Eloi Perdereau}
%\date{}

\begin{document}
\maketitle

\section{Analyse théorique}

\subsection{Algorithme naïf}

\begin{algorithm}
\Entree{T, n}
\Sortie{Sous-séquence maximale}
\Deb{
$S_{max} \leftarrow -\infty$ \\
\PourTous{$1 \leq k \leq n$}{
\PourTous{$k \leq l \leq n$}{
$S \leftarrow 0$ \\
\PourTous{$k \leq j \leq l$}{
$S \leftarrow S + T[j]$
}
\Si{$S > S_{max}$}{$S_{max} \leftarrow S$}
}
}
\Retour{$S_{max}$}
}
\caption{Naïf}
\end{algorithm}
\paragraph{Complexité}
\[O(n^3)\]

\subsection{Algorithme moins naïf}

\begin{algorithm}
\Entree{T, n}
\Sortie{Sous-séquence maximale}
\Deb{
$S_{max} \leftarrow -\infty$ \\
\PourTous{$1 \leq k \leq n$}{
$S \leftarrow 0$ \\
\PourTous{$k \leq l \leq n$}{
$S \leftarrow S + T[k]$ \\
\Si{$S > S_{max}$}{$S_{max} \leftarrow S$}
}
}
\Retour{$S_{max}$}
}
\caption{Moins naïf}
\end{algorithm}
\paragraph{Complexité}
\[O(n^2)\]

\subsection{Algorithme diviser pour régner}

\begin{algorithm}
\Entree{T, k, l}
\Sortie{Sous-séquence maximale}
\Deb{

\lSi{$l-k = 1$}{\Retour{$T[k]$}} \\
\lSi{$l-k = 2$}{\Retour{max\{$T[k]$, $T[k+1]$, $T[k]+T[k+1]$\}}}

$j \leftarrow \frac{l - k}{2}$ \\
$S_1 \leftarrow Diviser\_pour\_regner (T, k, k+j-1)$ \\
$S_2 \leftarrow Diviser\_pour\_regner (T, k+j+1, l)$ \\
$S_3 \leftarrow -\infty$ \\
$S_4 \leftarrow -\infty$ \\
$S_{tmp} \leftarrow T[j]$ \\
\Pour{i variant de j-1 à k descendant}{
$S_{tmp} \leftarrow S_{tmp} + T[i]$ \\
\Si{$S_{tmp} > S_3$}{$S_3 \leftarrow S_{tmp}$}
}
$S_{tmp} \leftarrow T[j]$ \\
\Pour{i variant de j+1 à l montant}{
$S_{tmp} \leftarrow S_{tmp} + T[i]$ \\
\Si{$S_{tmp} > S_3$}{$S_4 \leftarrow S_{tmp}$}
}
$S_0 \leftarrow S_3 + S_4 - T[j]$ \\
\Retour{$\max{S_0, S_1, S_2}$}
}
\caption{Diviser pour régner}
\end{algorithm}
\paragraph{Complexité}
\[O(n \log{n})\]

\subsection{Algorithme incrémental}

\begin{algorithm}
\Entree{T, n}
\Sortie{Sous-séquence maximale}
\Deb{
$S \leftarrow T[1]$ \\
$queue \leftarrow 1$ \\
\PourTous{$2 \leq i \leq n$}{
\eSi{$T[i] > 0$}{
$S_0 \leftarrow 0$ \\
\Pour{j variant de i-1 à queue+1 descendant}{
$S_0 \leftarrow S_0 + T[j]$
}
\Si{$S_0 > 0$}{
$queue \leftarrow i$
$S \leftarrow S + S_0$
} % fin si
} % fin si
{ % else
\Si{$S < T[i]$}{$S \leftarrow T[i]$}
} % fin else
}
}
\caption{Incrémental}
\end{algorithm}
\paragraph{Complexité}
\[O(n)\]

\end{document}
