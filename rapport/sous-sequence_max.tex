\documentclass[11pt, a4paper]{article}

\usepackage[english,francais]{babel}
\usepackage[utf8]{inputenc}
\usepackage[T1]{fontenc}
\usepackage[pdftex]{graphicx}
\usepackage{setspace}
\usepackage{hyperref}
\usepackage[french]{varioref}
\usepackage{amsfonts}
\usepackage{amssymb}
\usepackage[french,ruled]{algorithm2e}


\title{TP1 - Sous-séquences maximales}
\author{Sébastien Delecraz \and \'Eloi Perdereau}
%\date{}


\begin{document}

\maketitle

\begin{abstract}
  Ce rapport propose une étude de différents algorithmes pour résoudre le
  problème du calcul de la sous-séquence de somme maximale dans un tableau
  d'entier. Quatre algorithmes de complexité différentes sont présentés ici.
  On trouvera donc pour chacun leur pseudo-code, une étude de leur
  complexité et des tests de performance. Enfin nous confronterons les
  différents résultats obtenus et l'analyse théorique réalisée.
\end{abstract}

\section{Analyse théorique}

\subsection{Algorithme naïf}

\begin{algorithm}
  \Entree{T, n}
  \Sortie{Sous-séquence maximale}
  \Deb{
    $S_{max} \leftarrow -\infty$ \\
    \PourTous{$1 \leq k \leq n$}{
      \PourTous{$k \leq l \leq n$}{
        $S \leftarrow 0$ \\
        \PourTous{$k \leq j \leq l$}{
          $S \leftarrow S + T[j]$
        }
        \Si{$S > S_{max}$}{$S_{max} \leftarrow S$}
      }
    }
    \Retour{$S_{max}$}
  }
  \caption{Naïf}
\end{algorithm}
\paragraph{Complexité}
\[O(n^3)\]

\subsection{Algorithme moins naïf}

\begin{algorithm}
  \Entree{T, n}
  \Sortie{Sous-séquence maximale}
  \Deb{
    $S_{max} \leftarrow -\infty$ \\
    \PourTous{$1 \leq k \leq n$}{
      $S \leftarrow 0$ \\
      \PourTous{$k \leq l \leq n$}{
        $S \leftarrow S + T[k]$ \\
        \Si{$S > S_{max}$}{$S_{max} \leftarrow S$}
      }
    }
    \Retour{$S_{max}$}
  }
  \caption{Moins naïf}
\end{algorithm}
\paragraph{Complexité}
\[O(n^2)\]

\subsection{Algorithme diviser pour régner}

\begin{algorithm}
  \Entree{T, k, l}
  \Sortie{Sous-séquence maximale}
  \Deb{
    \lSi{$l-k = 1$}{\Retour{$T[k]$}} \\
    \lSi{$l-k = 2$}{\Retour{max\{$T[k]$, $T[k+1]$, $T[k]+T[k+1]$\}}}
    $j \leftarrow \frac{l - k}{2}$ \\
    $S_1 \leftarrow Diviser\_pour\_regner (T, k, j)$ \\
    $S_2 \leftarrow Diviser\_pour\_regner (T, j+1, l)$ \\
    $S_3 \leftarrow S_{tmp} \leftarrow T[j]$ \\
    \Pour{i variant de j-1 à k descendant}{
      $S_{tmp} \leftarrow S_{tmp} + T[i]$ \\
      \Si{$S_{tmp} > S_3$}{$S_3 \leftarrow S_{tmp}$}
    }
    $S_4 \leftarrow S_{tmp} \leftarrow T[j]$ \\
    \Pour{i variant de j+1 à l-1 montant}{
      $S_{tmp} \leftarrow S_{tmp} + T[i]$ \\
      \Si{$S_{tmp} > S_4$}{$S_4 \leftarrow S_{tmp}$}
    }
    $S_0 \leftarrow S_3 + S_4 - T[j]$ \\
    \Retour{max\{$S_0$, $S_1$, $S_2$\}}
  }
  \caption{Diviser pour régner}
\end{algorithm}
\paragraph{Complexité}
\[O(n \log{n})\]


\subsection{Algorithme incrémental}

\begin{algorithm}
  \Entree{T, n}
  \Sortie{Sous-séquence maximale}
  \Deb{
    $S_{max} \leftarrow T[1]$ \\
    $S_1 \leftarrow S_2 \leftarrow 0$ \\
    \PourTous{$2 \leq i \leq n$}{
      $S_1 \leftarrow S_1 + T[i]$ \\
      $S_2 \leftarrow S_2 + T[i]$ \\
      \Si{$S_2 < 0$}{
        $S_2 \leftarrow 0$
      }
      \Si{$S_1 > 0$}{
        $S_{max} \leftarrow S_{max} + S_1$ \\
        $S_1 \leftarrow S_2 \leftarrow 0$
      }
      \Si{$S_2 > S_{max}$}{
        $S_{max} \leftarrow S_2$ \\
        $S_1 \leftarrow S_2 \leftarrow 0$
      }
    }
    \Retour{$S_{max}$}
  }
  \caption{Incrémental}
\end{algorithm}
\paragraph{Complexité}
\[O(n)\]

\end{document}
